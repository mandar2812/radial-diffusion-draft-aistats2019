\documentclass[twoside]{article}

%\usepackage{aistats2019}
% If your paper is accepted, change the options for the package
% aistats2019 as follows:
%
\usepackage[accepted]{aistats2019}
\usepackage[utf8]{inputenc} % allow utf-8 input
\usepackage[T1]{fontenc}    % use 8-bit T1 fonts
\usepackage{hyperref}       % hyperlinks
\usepackage{url}            % simple URL typesetting
\usepackage{booktabs}       % professional-quality tables
\usepackage{amsmath}
\usepackage{amsfonts}       % blackboard math symbols
\usepackage{nicefrac}       % compact symbols for 1/2, etc.
\usepackage{microtype}      % microtypography
\usepackage{graphicx}
\usepackage{subfigure}
\usepackage{caption}
\bibliographystyle{apalike}
%
% This option will print headings for the title of your paper and
% headings for the authors names, plus a copyright note at the end of
% the first column of the first page.

% If you set papersize explicitly, activate the following three lines:
%\special{papersize = 8.5in, 11in}
%\setlength{\pdfpageheight}{11in}
%\setlength{\pdfpagewidth}{8.5in}

% If you use natbib package, activate the following three lines:
\usepackage[round]{natbib}
\renewcommand{\bibname}{References}
\renewcommand{\bibsection}{\subsubsection*{\bibname}}

% If you use BibTeX in apalike style, activate the following line:
%\bibliographystyle{apalike}

\begin{document}

% If your paper is accepted and the title of your paper is very long,
% the style will print as headings an error message. Use the following
% command to supply a shorter title of your paper so that it can be
% used as headings.
%
%\runningtitle{I use this title instead because the last one was very long}

% If your paper is accepted and the number of authors is large, the
% style will print as headings an error message. Use the following
% command to supply a shorter version of the authors names so that
% they can be used as headings (for example, use only the surnames)
%
%\runningauthor{Surname 1, Surname 2, Surname 3, ...., Surname n}

\twocolumn[

\aistatstitle{Bayesian Inference of Plasma Diffusion Parameters: Learning from Irregular \& Sparse Data}

\aistatsauthor{Anonymous}

\aistatsaddress{Anonymous} ]

\begin{abstract}
  We present a novel method which combines sparse irregular observations of a field with its governing 
  physical dynamics, for performing Bayesian inference over latent system parameters. Our method uses 
  a basis function approach coupled with a least squares support vector machine objective function which weighs
  differently errors arising due to data fitting and satisfaction of physical constraints. The method is applicable to linear PDE systems, it incorporates physical models into classical least squares techniques for the purpose of
  data assimilation and uncertainty quantification of latent parameters. We apply this method to the inverse
  problem of inferring uncertainty estimates of plasma diffusion parameters from sparse observations.
\end{abstract}

\section{Introduction}

The Earth's radiation belts are the regions of space near 
the Earth that extend between 2 and 8 Earth's radii, where the terrestrial magnetic field traps 
electrons and ions in complex electromagnetic orbits  \citep{vanAllen}. Since their discovery,
the belts have been the subject of intensive research due to their complex behavior
and damaging effects on spacecraft \citep{GUBBY20021723, WellingSatellite, baker2002}.

Radiation belt particles generally execute three types of periodic motion, 
each with its own corresponding adiabatic invariant: 
gyration about magnetic field lines, bounce along field lines, 
and drift around the Earth. During active times, when conditions 
change on time scales shorter than the periods of motion, 
adiabaticity can be broken and particle motion can not be
simply decomposed into the aforementioned components. Particle motion 
can then be represented diffusively along each component via the 
\emph{Fokker-Planck} equation yielding a powerful picture of 
radiation belt dynamics \citep{schulz2012particle}.

The third invariant represents the total magnetic flux enclosed within 
a full particle orbit. It is common to use a normalized form of this, the 
so called Roederer $L^{*}$ \citep{Roederer1970}. It is analogous to radial 
distance from the center of the Earth (in Earth radii) to the equatorial 
crossing point of the bouncing particle. Diffusion in $L^{*}$ alone 
(the other invariants shall be considered conserved) accounts for the
capture and inward radial transport of radiation belt particles 
(\cite{Roederer1970}, \citep{JGR:JGR4463}).

One of the main difficulties of using a physics-based model for studying and 
forecasting energetic electrons in the radiation belt is that the parameters 
that characterize the Fokker-Planck equation, namely diffusion tensor and loss term,
are not directly observable. Hence, their determination represents an inverse problem,
which is generally difficult to solve and can often become ill-posed.

In this paper we propose an inference model which can learn from sparse
data while taking into account prior knowledge of the system dynamics
in the form of a linear partial differential equation. The method replaces
the finite difference solver with a surrogate model which tries to fit
the observations and the system dynamics. The surrogate is expressed as
a basis function expansion whose coefficients are computed by formulating
a \emph{Least Squares Support Vector Machine}(LSSVM) like optimization
objective.

In the proceeding sections, we give a short introduction to the radial 
diffusion equation used in magnetospheric physics. After an overview of 
the parameterizations of the radial diffusion unknowns used by the research community, 
we give a detailed formulation of our proposed method and demonstrate 
how one may use it for performing inference over said diffusion parameters.

\subsection{Plasma Diffusion}

The radial diffusion system is a simplified one-dimensional version of
the \emph{Fokker-Planck} equation. It tracks the time evolution of the
\emph{phase space density} of particles, $f$ which is governed by the
differential equation (\ref{eq:raddiffusion}) known as \emph{radial
  diffusion} in the radiation belt community\citep{JGRA:JGRA9345}.

\begin{equation}\label{eq:raddiffusion}
  \frac{\partial{f}}{\partial{t}} = l^2 \frac{\partial}{\partial{l}}\left( \frac{\kappa(l,
      t)}{l^{2}} \frac{\partial{f}}{\partial{l}} \right) - \lambda(l,
  t) f +  Q(l, t)
\end{equation}

The \emph{phase space density} $f$ is a function of the spatial
coordinate $l$ which denotes the \emph{Roederer} $L^*$ or
\emph{L-shell}, and time $t$.

The key quantities in the system above are.

\begin{enumerate}
\item $f$: The density of particles as a function of space $l$ and
  time $t$.
\item $\kappa(l, t)$: Diffusion field.
\item $\lambda(l, t)$: Loss rate, this is a non-negative
  quantity which indicates how quickly particles are lost from the
  radiation belts.
\item $Q(l, t)$: Particle injection rate.
\end{enumerate}

\subsubsection*{Diffusion Parameters}

To solve the radial diffusion system \ref{eq:raddiffusion}, the
quantities $\kappa(l, t)$, $\lambda(l, t)$ and $Q(l, t)$ need to be
specified. It is a common practice (see \citet{GRL:GRL10762},
\citet{JGRA:JGRA15067}, \citet{JGRA:JGRA18021} and
\citet{GRL:GRL22815}) to parameterise the diffusion field
$\kappa$ and loss rate $\lambda$ in the following manner.

\begin{align}
  \kappa(l,t), \lambda(l, t) \sim \alpha l^{\beta} 10^{b Kp(t)} \\
\end{align}

The quantities $\alpha$, $\beta$ and $b$ are parameters which define
the diffusion field and loss rate while the quantity $Kp(t)$ is known
as the Kp index, a measured quantity which stands as a proxy for the
global geomagnetic activity \cite{BartelsKp}.


\section{Inverse Problems}\label{sec:inv}

The \emph{inverse problem} can be stated as follows: given a set of
noisy observations $y$ scattered in the space time domain, of a
physical quantity $f$ governed by the dynamical system
$\mathcal{L}_{\theta} f = Q_{\theta}$, estimate the parameters $\theta$ of the
dynamical system $(\mathcal{L}_{\theta}, Q_{\theta})$.

In this formalism $\mathcal{L}_{\theta}$ is a differential operator,
$Q_{\theta}$ is a \emph{source term} and $\theta$ is a collection of parameters
which specify the operator and the source term.

In the radial diffusion system (\ref{eq:raddiffusion}), $\mathcal{L}_{\theta} =
\frac{\partial}{\partial{t}} - l^2 \frac{\partial}{\partial{l}}\left( \frac{\kappa(l,
      t)}{l^{2}} \frac{\partial}{\partial{l}} \right) + \lambda(l,
  t)$. In this case $\theta$ would be a collection of parameters which
  would specify the analytic expressions for $\kappa(l,t)$,
  $\lambda(l,t)$ and $Q(l,t)$.


\subsection{Related Work}

\subsubsection*{Meshfree PDE solutions}

\paragraph{Least Squares Support Vector Machines}

(LSSVM) have been applied to calculating approximate solutions to PDEs
\citep{MEHRKANOON2015105}, \citep{MEHRKANOON20122502} as well as
parameter estimation of delay differential equations
\citep{MEHRKANOON2014830}, the approach taken in the aforementioned
research (\citet{MEHRKANOON2014830}) was expressing the parameter
estimation of time delay as an algebraic optimization problem
resulting in closed-form approximation for the time varying parameters
while avoiding iterative simulation of the dynamical system (governed
by the delay differential equations) in the parameter estimation process.

\paragraph{Radial Basis Functions}

(RBF) were first applied for solution of PDE problems in \cite{KANSA1990147}, 
the authors used colocation with \emph{multiquadric} basis functions for approximating solutions
of boundary value problems.

Radial basis functions have been applied for the mesh-free solutions of Poisson PDE systems 
(\citep{AMINATAEI20082887}, \citep{DUAN200866}, \citep{DUAN2006394} \& \citep{CNM:CNM419}),
as well as the Poisson control problem \citet{Pearson2013}. Further applications of RBFs include 
atmospheric flow \citep{Tillenius2015406}, convection-diffusion \citep{Safdari-Vaighani2015}, and 
Schrödinger's equation (\citep{doi:10.1137/120893975, doi:10.1063/1.3637863}). See ref. \citep{fornberg2015} for a recent textbook with geoscience applications.

\paragraph{Gaussian Processes}

Gaussian Process (GP) models \citep{Rasmussen:2005:GPM:1162254} have a rich theory which has much overlapping with linear systems and deterministic and stochastic differential equations. 

\citet{Skilling1992} presented one of the earliest works which focused on calculating solutions of ordinary differential equations (ODE) systems with Gaussian Process methodology, \citet{Graepel} applied it for solving linear partial differential equations with Dirichlet and Von Neumann boundary conditions.

Interplay between linear operators and GP models applied to Bayesian filtering was investigated by \citet{Sarkka2011}. \citet{pmlr-v31-dondelinger13a} proposed an adaptive gradient matching technique to used Gaussian Process models for infering parameters of coupled ODE systems.

\paragraph{Neural Networks} were also employed for solutions of boundary value problems in the works such as \citet{Lagaris}, \citet{Aarts2001} \&  \citet{TSOULOS20092385}. \citet{Baymani2011} used Feedforward networks for calculating solutions to the Stokes problem. These approaches generally revolved around decomposing the solution into two components, i.e. the first one satisfying the boundary conditions and the second one represented by the feedforward network.

\section{Methodology}

Performing Bayesian inference over parameters of physical systems,
involves synthesizing pre-exsiting knowledge of the physical system in
question i.e. the \emph{partial differential equation} (PDE), with
statistical techniques. The aim of such an exercise is often the
quantification of uncertainty over system parameters from an often
sparse set of observations which are quantities of interest in the
physical system. 


\subsection{Model Formulation}

We approach the radial diffusion inference problem by formulating a
modified version of the \emph{least squares support vector machine}
predictor for obtaining a closed form approximation to the phase space
density $f$ which tries to satisfy the radial diffusion PDE
\ref{eq:raddiffusion} on a fixed set of \emph{colocation} points while
minimizing error on a set of sparse noisy observations. 
Using the surrogate phase space density estimator as a baseline, we
specify the likelihood of the observations.


\subsubsection*{Surrogate Phase Space Density Model}

Let $\mathcal{D}={(x^{o}_{i}, y_{i}): i = 1 \cdots n_{o}}$ be a set of
noisy observations of the phase space density $f$, where $x_{i} =
(l_{i}, t_{i})$ are points in the space time domain. We seek a linear
estimator for $f$ of the form $\hat{f}(x) = w^{T}\varphi(x) + b$,
where $\varphi(.): \mathbb{R}^{2} \rightarrow \mathbb{R}^{d}$ is a $d$
dimensional feature map and $b$ is a scalar intercept.

Further let $\mathcal{C} ={(x^{c}_{i}, q_{i}): i = 1 \cdots n_{c}}$ be 
a set of colocation points on which we aim to enforce radial diffusion
dynamics. The values $q_{i}$ represent the particle injection rate $Q$ at $x^c$ and 
are calculated each time the parameters of Q are sampled, by evaluating the expression 
$Q(l,t) = (\alpha_{Q}l^{\beta_{Q}} + \gamma_{Q})10^{b_{Q}Kp(t)}$.

We exploit the linearity of the differential operator
$\mathcal{L}_{\theta}$ and note that $\mathcal{L}_{\theta} [\hat{f}(x)]
= w^{T} \mathcal{L}_{\theta}[\varphi(x)]$, yielding an estimator
$\hat{Q}(x) = w^{T}\psi(x)$ where $\psi_{\theta}(x) =
\mathcal{L}_{\theta}[\varphi(x)]$. Calculating $w \in \mathbb{R}^d$
can now be cast as the following constrained $L_2$ regularized 
least squares problem.

\begin{align}\label{eq:surrogate}
   & min_{w,e,\epsilon} \ \mathcal{J}(w,e,\epsilon;\theta) = \\
   & \frac{1}{2} w^{T}w + \frac{1}{2\gamma_{o}} \sum_{k = 1}^{n_{o}}{e^{2}_{k}} + \frac{1}{2\gamma_{c}} \sum_{k = 1}^{n_{c}}{u_{k} \epsilon^{2}_{k}} \\
  s.t &\nonumber \\
  & y_{i}  = w^{T}\varphi(x^{o}_{i}) + b + e_{i}, \ \ i = 1 \cdots n_{o} \\
  & q_{i} = w^{T}\psi_{\theta}(x^{c}_{i}) + \epsilon_{i}, \ \ i = 1 \cdots n_{c}
\end{align}

It can be seen that system \ref{eq:surrogate} is similar to the
formulation of the LSSVM model, while incorporating the dynamics of
linear PDE systems into its loss function. 

The quantities $\gamma_{o}$ and $\gamma_{c}$ are weights attached to
the errors on observations and colocation points respectively, thus
by smoothly varying them one may assign higher or lower importance for
the surrogate model to fit the observational data and the dynamics of
the physical system. The quantities $u_i$ enable us to weigh each colocation
point differently.

In order to solve this system one must construct its
\emph{Lagrangian}.

\begin{align*}\label{eq:lag}
      & \mathfrak{L}(w,e,\epsilon, \alpha_{1 \cdots k}, \beta_{1 \cdots k}; \theta; \gamma_{o}; \gamma_{c}) = \\ 
      & \frac{1}{2} w^{T}w + \frac{1}{2\gamma_{o}} \sum_{k = 1}^{n_{o}}{e^{2}_{k}} +
      \frac{1}{2\gamma_{c}} \sum_{k = 1}^{n_{c}}{u_{k} \epsilon^{2}_{k}} \\
      & + \sum_{k = 1}^{n_{o}}{\alpha_{k}(y_{k} - w^{T}\varphi(x^{o}_{k}) - b - e_{k})} \\
      & + \sum_{k = 1}^{n_{c}}{\beta_{k} (q_{j} - w^{T}\psi_{\theta}(x^{c}_{j}) - \epsilon_{j})} 
\end{align*}

Given fixed values for $\gamma_{o}$, $\gamma_{c}$ and PDE parameters
$\theta$, the equation above expresses the Lagrangian of system
\ref{eq:surrogate}. The quantities $\alpha_{1}, \cdots, \alpha_{n_{o}}$ and
$\beta_{1}, \cdots, \beta_{n_{c}}$ are the \emph{Lagrange multipliers}
introduced for equality constraints of the system. Applying the
\emph{Karush-Kuhn-Tucker}(KKT) conditions the solution of the
optimization problem \ref{eq:surrogate} can be expressed in terms of
the Lagrange multipliers $\alpha = (\alpha_{1}, \cdots, \alpha_{n_{o}})$
$\beta = (\beta_{1}, \cdots, \beta_{n_{c}})$.

\begin{equation}\label{eq:solution}
  \begin{bmatrix}
    0 & \mathbf{1}^{T} & \mathbf{0} \\ 
    \mathbf{1} & \Omega + \gamma_{o}I  & \Omega_*\\ 
    \mathbf{0} & \Omega_{*}^{T}  & \Omega_{**} + \gamma_{c}U 
  \end{bmatrix} \begin{bmatrix}
    b\\ 
    \alpha\\ 
    \beta
  \end{bmatrix} = \begin{bmatrix}
    0\\ 
    y\\ 
    q
  \end{bmatrix}
\end{equation}

The components of the symmetric block matrix system on the left hand side of 
\ref{eq:solution} are 
\begin{enumerate}
\item $\Omega \in \mathbb{R}^{n_{o} \times n_{o}}: \omega_{ij} = \varphi(x^{o}_{i})^{T} \varphi(x^{o}_{j})$
\item $\Omega_{**} \in \mathbb{R}^{n_{c} \times n_{c}}: \omega^{**}_{ij} = \psi(x^{c}_{i})^{T} \psi(x^{c}_{j})$
\item $\Omega_{*} \in \mathbb{R}^{n_{o} \times n_{c}}: \omega^{*}_{ij} = \varphi(x^{o}_{i})^{T} \psi(x^{c}_{j})$
\item $U \in \mathbb{R}^{n_{c} \times n_{c}} = \begin{bmatrix}
    u_1 & \cdots & 0 \\ 
    \vdots & \ddots  & 0\\ 
    0 & \cdots  & u_{n_{c}} 
  \end{bmatrix}$
\end{enumerate}

The surrogate model can now be used to estimate the phase space density 
at a point $x = (l,t)$.

\begin{equation}\label{eq:model}
\hat{f}(x;\theta) = \sum_{k = 1}^{n_{o}}{\alpha_{k}\varphi(x)^{T}\varphi(x^{o}_{k}) + \sum_{k = 1}^{n_{c}}}{\beta_{k} \varphi(x)^{T} \psi_{\theta}(x^{c}_{k})} + b
\end{equation}

\subsubsection*{Choice of Basis}

There exist several choices regarding the basis $\varphi(.)$, they are but not limited to 
orthogonal polynomials, Fourier series, radial basis functions etc. 

For our problem we choose a basis which is a product of a Chebyshev basis of maximum degree 10 in space and a Laguerre basis of maximum degree 4, in time.

\begin{equation}\label{eq:basis}
\varphi_{i,j}(l,t) = C_{i}(l) L_{j}(t)
\end{equation}


\subsubsection*{Role of $\gamma_o$, $\gamma_c$ and $u_i$}

The quantities $\gamma_o$ and $\gamma_c$ serve to control the importance assigned to
of errors made on the observations and colocation points respectively. Varying them 
gives the modeler the ability to vary the behavior of the surrogate model. In the 
limiting case of $\gamma_c$ tending to zero, the model behaves as if the PDE dynamics
is enforced as a hard constraint. This case is equivalent to the following formulation.

\begin{align}\label{eq:surrogate2}
   min_{w,e} \ \mathcal{J}(w,e,\epsilon;\theta) &= 
   \frac{1}{2} w^{T}w + \frac{1}{2\gamma_{o}} \sum_{k = 1}^{n_{o}}{e^{2}_{k}} \\
  & s.t \nonumber \\
  y_{i} & = w^{T}\varphi(x^{o}_{i}) + b + e_{i}, \ \ i = 1 \cdots n_{o} \\
  q_{i} & = w^{T}\psi_{\theta}(x^{c}_{i}), \ \ i = 1 \cdots n_{c}
\end{align}

Although choosing $\gamma_c = 0$ is an appropriate choice if one wishes
to enforce the physical dynamics as a constraint, it can possibly lead to numerical
instabilities in inverting system \ref{eq:solution} and hence choosing a non zero value 
for $\gamma_c$ works better in practice.

For the experiment outlined in \ref{sec:exp}, we choose $\gamma_o = \gamma_c = 10$.

The weights $u_i$ have a special interpretation in a particular context. It is possible to
interpret the term $\sum_{k = 1}^{n_{c}}{u_{k} \epsilon^{2}_{k}}$ as a quadrature approximation
to the integrated error of the surrogate model with respect to the governing dynamics $\int_{x \in \mathcal{D}}{||\mathcal{L}_{\theta} [\hat{f}(x)] - Q(x)||^2}$. When
$u_i = \frac{1}{n_c}$, this corresponds to the \emph{Monte Carlo} quadrature of 
the integrated error, but it is possible to improve the quadrature accuracy by using \emph{Gauss-Legendre} quadrature.

In section \ref{sec:exp}, we use eight point \emph{Gauss Legendre} quadrature in space and time dimension each, thereby setting $n_c = 64$ and weights $u_i$ to appropriate values as dictated by the quadrature rule (\cite{_abramowitzm}).

\subsection{Quantifying Observation Likelihood}

We assume a multivariate Gaussian distribution for calculating the likelihood of 
the observations conditioned on the system parameters $\theta$. 

The surrogate model \ref{eq:model} gives a baseline or mean value for the phase space density, we use a
hybrid RBF covariance function $C(x_{i}, x_{j}) = \sigma^2 exp(-\frac{1}{2} (\frac{|l_i - l_j|^2}{s} + \frac{|t_i - t_j|}{r}))$
to quantify the covariance of the phase space density $f$ over two points 
$x_i = (l_i, t_i)$ and $x_j = (l_j, t_j)$ in the domain. 

\begin{equation}
\mathbf{y} | x_1, \cdots, x_{n_o}, \theta \sim \mathcal{N} \left(\mathbf{\mu}_f, \Sigma \right )
\end{equation}

\begin{equation}
\mathbf{y} = \begin{bmatrix}
y_1\\ 
\vdots\\ 
y_{n_o}
\end{bmatrix}
\end{equation}

\begin{equation}
  \mathbf{\mu}_f = \begin{bmatrix}
\hat{f}(x_1)\\ 
\vdots\\ 
\hat{f}(x_{n_o})
\end{bmatrix}
\end{equation}

\begin{equation}
    \Sigma = \begin{bmatrix}
C(x_1, x_1) & \cdots  & C(x_1, x_{n_o})\\ 
\vdots & \ddots & \vdots\\ 
C(x_{n_o}, x_{n_{1}}) & \cdots  & C(x_{n_o}, x_{n_{o}})
\end{bmatrix}
\end{equation}

The values of $s$ and $r$, the length scales of the covariance function, 
can be fixed to the size of the space-time grid of the radial basis functions, 
alternatively they can also be treated as system parameters which can be sampled 
by the inference procedure. Since the core aims of this research was the quantification
of the uncertainty over the parameters of the radial diffusion system, 
we treat the covariance function parameters as fixed.

\subsection{Inference}

We employ the adaptive Metropolis algorithm as proposed by \citet{haario2001}, for sampling
system parameters. The adaptive Metropolis algorithm adapts the exploration variance according
to the running sample statistics of the Markov Chain procedure.


\section{Experiment}\label{sec:exp}

For the purposes of the experiment, the parameters of $\kappa$ and $\lambda$ fixed while MCMC inference 
is performed on the parameters of $Q$. The prior distributions chosen for the parameters 
are shown in table \ref{tab:prior}. The posterior distribution over the parameters of $\lambda$ is 
sampled via adaptive Metropolis, the first 5000 samples are kept aside as the \emph{burn in} period
of the Markov Chain.

\begin{table}[t]
  \caption{Parameters: Prior}
  \label{tab:prior}
  \centering
  \begin{tabular}{ll}
    \toprule
    \cmidrule{1-2}
    Parameter     & Prior\\
    \midrule
    $\alpha$ & $Lognormal(0, 1)$ \\
    $\beta$  & $Lognormal(1, 1)$ \\ 
    $\gamma$ & $Lognormal(0, 1)$ \\ 
    $b$ & $\mathcal{N}(0, 1)$ \\
    \bottomrule
  \end{tabular}
\end{table}


\subsection*{Data Generation}

The technique presented is applied on synthetic data generated using a radial diffusion solver, 
the ground truth values of the radial diffusion parameters are listed in table \ref{tab:ground-truth}.

A slightly modified parameterization is adopted for the particle injection $Q$ as shown in equation \ref{eq:q} below. The particle injection has a strictly time varying component due to presence of a constant $\gamma$.

\begin{equation}\label{eq:q}
Q(l,t)  \sim (\alpha l^{\beta} + \gamma) 10^{b Kp(t)}
\end{equation}

The initial phase space density $f(t = 0)$ and the time trajectory 
of the Kp index are assumed to be set as follows.

\begin{align}
f(t = 0) &= 4000 + 1000(C_{3}(l_*) - C_{5}(l_*)) \\
l_* &= 2\frac{l - l_{min}}{l_{max} - l_{min}} - 1 \\
Kp(t) &= \left\{\begin{matrix}
2.5 + 4t & 0 \leq t < 1.5\\ 
8.5 & 1.5 < t \leq 3\\ 
17.5-3t & 3 <  t \leq 5 \\ 
2.5 & 5 < t\\ 
\end{matrix}\right.
\end{align}

Where $C_n(.)$ is the is the Chebyshev polynomial of degree n. 
The evolution of the Kp index \citep{BartelsKp} is assumed such that it mimics a geomagnetic storm event. 

The radial diffusion solver is run for domain limits $l \in [1, 7], t \in [0, 5]$ with
200 bins in the spatial and 50 bins in the temporal domains respectively.

After the approximate solution profiles of $f$ are generated, the points are sub-sampled
uniformly such that 100 points lie in the interior of the domain and 30 points at the initial time step (t = 0) are selected. These observations are then perturbed by Gaussian noise to yield
the final observation set $\mathcal{D}$ which is fed, as training data to the surrogate model $\hat{f}(x)$.

\subsection*{Results}

The posterior inferred for parameters of the particle injection rate $Q$ \ref{fig:alpha}, 
\ref{fig:beta}, \ref{fig:gamma} and \ref{fig:b} respectively and the samples from each prior distribution are shown in \ref{fig:alphaprior}, \ref{fig:betaprior}, \ref{fig:gammaprior} and \ref{fig:bprior} respectively.

We see that the marginal posterior distributions of each parameter have high probability 
density near the ground truth, and that they are heavy tailed because there are often large 
regions of the parameter space which produce similar solutions for the phase space density $f$. 
Apart from identifying of parameters, the model also helps in significant reduction of uncertainty
from prior to posterior for the parameters $\beta$, $\gamma$ and $b$.

The strength of the proposed method is the ability to quantify the uncertainty in diffusion parameters from a sparse set of observations. Due to the formulation of the surrogate as a dual optimization problem, it allows the inference to scale well with respect to high dimensional basis function expansions. The method shows promise for parameter inference of physical systems and warrants further research in its improvement.


\begin{table}[t]
  \caption{Parameters: Ground Truth}
  \label{tab:ground-truth}
  \centering
  \begin{tabular}{lllll}
    \toprule
    \cmidrule{1-2}
    Quantity     & $\alpha$     & $\beta$ & $\gamma$ & $b$ \\
    \midrule
    $Q$ & $1$  & $0.5$ & $0.05$  & $0.45$     \\
    $\kappa$  & $4.731 \times 10^{-10}$ & $10$  & $0$ & $0.506$ \\
    $\lambda$ & $0.3678$ & $0.5$  & $0$ & $-0.2$ \\
    \bottomrule
  \end{tabular}
\end{table}

\begin{figure}[h]
\vspace{.3in}
\centerline{\includegraphics[width=0.5\textwidth]{histogram_Q_alpha_posterior.png}}
\vspace{.3in}
\caption{Posterior samples for $\alpha$ parameter of Q, red dotted line indicates the ground truth}
\label{fig:alpha}
\end{figure}

\begin{figure}[h]
\vspace{.3in}
\centerline{\includegraphics[width=0.5\textwidth]{histogram_Q_alpha_prior.png}}
\vspace{.3in}
\caption{Prior samples for $\alpha$ parameter of Q, red dotted line indicates the ground truth}
\label{fig:alphaprior}
\end{figure}

\begin{figure}[h]
\vspace{.3in}
\centerline{\includegraphics[width=0.5\textwidth]{histogram_Q_beta_posterior.png}}
\vspace{.3in}
\caption{Posterior samples for $\beta$ parameter of Q, red dotted line indicates the ground truth}
\label{fig:beta}
\end{figure}

\begin{figure}[h]
\vspace{.3in}
\centerline{\includegraphics[width=0.5\textwidth]{histogram_Q_beta_prior.png}}
\vspace{.3in}
\caption{Prior samples for $\beta$ parameter of Q, red dotted line indicates the ground truth}
\label{fig:betaprior}
\end{figure}

\begin{figure}[h]
\vspace{.3in}
\centerline{\includegraphics[width=0.5\textwidth]{histogram_Q_gamma_posterior.png}}
\vspace{.3in}
\caption{Posterior samples for $\gamma$ parameter of Q, red dotted line indicates the ground truth}
\label{fig:gamma}
\end{figure}

\begin{figure}[h]
\vspace{.3in}
\centerline{\includegraphics[width=0.5\textwidth]{histogram_Q_gamma_prior.png}}
\vspace{.3in}
\caption{Prior samples for $\gamma$ parameter of Q, red dotted line indicates the ground truth}
\label{fig:gammaprior}
\end{figure}


\begin{figure}[h]
\vspace{.3in}
\centerline{\includegraphics[width=0.5\textwidth]{histogram_Q_b_posterior.png}}
\vspace{.3in}
\caption{Posterior samples for $b$ parameter of Q, red dotted line indicates the ground truth}
\label{fig:b}
\end{figure}

\begin{figure}[h]
\vspace{.3in}
\centerline{\includegraphics[width=0.5\textwidth]{histogram_Q_b_prior.png}}
\vspace{.3in}
\caption{Prior samples for $b$ parameter of Q, red dotted line indicates the ground truth}
\label{fig:bprior}
\end{figure}


\subsubsection*{Acknowledgments}

\medskip

\small

\bibliography{references}

\end{document}